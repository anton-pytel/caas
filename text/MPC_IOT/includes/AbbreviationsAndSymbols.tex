CRM - Customer Relationship Management - aplikácia na správu vzťahov so zákazníkom \\
ELK - Elasticsearch, Logstash, Kibana - sada nástrojov na prácu s Big Data \\
ERP - Enterprise Resource Planning - plánovanie zdrojov v podniku \\
ESB - Enterprise Service Bus - servisná zbernica podniku \\
GPIO - General purpose input/output \\
HTML - HyperText Markup Language - značkovací jazyk používaný na vytváranie www stránok \\
HTTP - HyperText Transfer Protocol - protokol na prenos štrukturovaných informácií. \\
IEEE - Institute of Electrical and Electronics Engineers - Inštitút elektrotechnických a elektronických inžinierov \\
IoT - Internet of Things - internet vecí \\
IT - Information Technology \\
JSON - JavaScript Object Notation - Notácia na definovanie Javascript objektov \\
KPIs - Key Performance Indicators - kľúčove ukazovatele výkonnosti \\
MIMO - multiple input multiple output - viacrozmerný systéme \\
MPC - Model predictive controller - prediktivný regulátor \\
MVC - Model-View-Controller štruktúra aplikácie \\
NoSQL - non SQL - nie relačná databáza \\
OT - Operationals Technology \\
PAN - Personal Area Network \\
PID - regulátor s Proporčnou, Integračnou a Derivačnou zložkou \\
RAM - Random access memory - operačná pamäť \\
REST - Representational state transfer architectural style \\
RFID - Radio Frequency IDentification - identifikácia pomocou rádiovej frekvencie \\
RPC - Remote Procedure Call - volanie vzdialenej procedúry \\
RSS - Rich Site Summary \\
SISO - single input single output - jednorozmerný systém \\
SOA - service oriented architecture - architektúra so službou ako základným prvkom \\
SOAP - Simple Object Access Protocol - protokol výmenu štrukturovanej informácie \\
TCP/IP - Transmission Contro Protocol/Internet protocol \\ 
TICK - Telegraf, InfluxDB, Chronograf, Kapacitor sada nástrojov na prácu s Big Data\\
UML - Unified Modeling Language - unifikovnaý modelovací jazyk \\
UPnP - Universal Plug aNd Play \\
VPN - Virtual Private Network \\
Wifi - bezdrôtové protokoly rodiny 802.11 \\
WSDL - Web Service Description Language - jazyak na opísanie webovej služby \\
WWW - World Wide Web - Celosvetová sieť (stránok s HTML obsahom) \\
XML - eXtensible Markup Language - rozšíriteľný značkovací jazyk
