\indent V súčasnosti existuje mnoho metód automatického riadenia systémov. Od klasických metód ako je PID regulátor, cez rôzne pokročilé metódy ako je robustné, adaptívne alebo aj prediktívne riadenie. Predmetom tejto práce je
\begin{itemize}
\item  spomenúť hlavné výhody prediktívnej metódy a jej súčasný stav.
\item  Predmetom práce je tiež definovať pojem Internet vecí (Internet of Things - IoT) ako nový typ architektúry informačných systémov a jeho špecifiká. 
\item  V praktickej časti je spojenie predošlých dvoch bodov v myšlienke regulátor ako služba (Controller as a Service - CaaS) a jej implementácia. 
\end{itemize}
 
\subsection*{Motivácia k MPC}
Prediktívne riadenie (MPC, model predictive controller) je pokročilá metóda riadenia založená na optimalizácii, ktorá bola využívaná na aplikáciu v systémoch s pomalou dynamikou, napríklad v chemických, či petrochemických procesoch. Na rozdiel od lineárno-kvadratického regulátora, MPC na optimálne riadenie ponúka explicitné ošetrenie procesných obmedzení, ktoré vznikajú z prirodzených požiadaviek, napríklad efektivita nákladov, bezpečnostné obmedzenia akčných členov a iné.\cite{MPC01} \newline
\indent Hlavná výhoda regulátora je, že je riešený ako optimalizačný problém, takže sa snaží minimalizovať okrem iného hlavne potrebný riadiaci zásah na dosiahnutie žiadaného výstupu riadeného systému. Ďalšia výhoda pri riešení optimalizačného problému je, že sa jednoducho môžu zadať ohraničenia systému. Rôzne obmedzenia môžu byť aplikované na riadený systém a napriek tomu môže riaditeľný s minimálnym riadiacim zásahom. Táto metóda riadenia môže byť prirovnaná k prepočítavaniu ťahov v šachu. Pri prepočítavaní sa ráta s tým, čo sa môže udiať v čase v závislosti od poznania procesu, pri šachovej hre podľa jej pravidiel a podľa toho optimalizovať riadiace zásahy, ťahy v šachu, aby sa dosiahol najlepší výsledok z dlhodobého hľadiska, v prípade šachu, to je vyhratá partia. Pri MPC regulátoroch sa dá predísť javom, ktoré sa vyskytujú pri konvenčných regulátoroch ako PID, ako sú riadiace zásahy, ktoré dosahujú dobré výsledky z krátkodobého hľadiska, ale v konečnom dôsledku sa prejavia ako vysoko nákladné. Tento jav môže byť pomenovaný ako ,,vyhrať bitku, ale prehrať vojnu``. \\ 
\indent Ďalšie výhody MPC regulátora sú také, že jednoducho vedia riadiť viac premenné systémy a ako už bolo spomenuté, zahrnúť obmedzenia systému – výstupu, riadiaceho zásahu, stavu systému, už pri návrhu regulátora. MPC regulátor obsahuje viacero premenných, pomocou, ktorých je možné ho vyladiť takmer pre každý proces. \\
\indent Medzi nevýhody MPC regulátora patrí napríklad to, že niektoré MPC modely sú limitované len na stabilný proces v otvorenej slučke. Často vyžadujú veľký počet koeficientov modelu na opis odozvy systému. Niektoré MPC modely zase sú formulované na  rušenie na výstupe a tie by ťažko mohli zvládnuť poruchy na vstupe. Niektoré MPC modely sú zase upravované na výstupe, pretože model nie je totožný s reálnym systémom. Tieto modely sú zvyčajne upravené o konštantu, podľa nameraných údajov, nerátajú však s tým, že táto zmena na výstupe sa môže v budúcnosti zmeniť, čo môže mať za následok, že finálny výsledok nebude optimálny. Taktiež ak horizont predikcie nie je zvolený správne, tak riadenie nebude optimálne aj keď model systému správny bude. Niektoré systémy majú širokú škálu prevádzkových podmienok, ktoré sa často menia. Medzi príklady patria exotermické reaktory, procesy na dávkové spracovanie a tiež systémy, kde rôzny spotrebitelia majú rôzne špecifikácie produktov. Lineárne modely MPC regulátorov nie sú schopné zvládnuť dynamické správanie týchto procesov, preto musí byť použitý nelineárny model pre lepšie riadenie.\cite{MPC02} 
\subsection*{Motivácia k IoT}
Aktuálne je pojem IoT čoraz častejšie skloňovaný na konferenciách akademickej a rovnako aj komerčnej sféry. Rôzne popredné spoločnosti ako IDC, Gartner ai. zaoberajúce sa výskumnými a poradnými činnosťami v oblasti informačných a komunikačných technológií robia odhady využitia. Tvrdenie portálu www.crn.com:
,,Napriek tomu, že IoT bol vždy trochu vágny pojem pri špecifikácii obchodných partnerov a produktov, ktoré už sú na trhu, analytici predikovali, že pripojené zariadenia (connected devices) majú veľký potenciál príležitosti, ktoré budú výnosné. Júnový prieskum trhu spoločnosti IDC predikoval, že výdavky na IoT dosiahnú v roku 2020 sumu vo výške 1,7 biliónov dolárov, zatiaľ čo Gartner predpovedal, že v tom istom roku bude pripojených 21 miliárd zariadení.``\cite{IOT01} Na základe tohto a ďalších podobných článkov je teda motivácia hľadanie vhodných prípadov využitia IoT.\\
\indent Okrem toho treba zopakovať, že IoT je spojenie minimálne dvoch rozsiahlych technických odborov, neberúc do úvahy spoločenské, prírodne ani lekárske vedné odbory, ktoré tiež môžu skúmať dopad IoT na ne. Rozsiahlosť tejto problematiky je určite nepopierateľná. Druhá motivácia teda je získavanie  nadhľadu nad spleťou vznikajúcich a zanikajúcich technológií a štandardov.\\
\indent Ostatná motivácia k skúmaniu IoT je porovnanie rozdielov pri návrhu, vývoji a správe oproti klasickým čisto softvérovým informačným systémom. 
\subsection*{Motivácia k CaaS}
Napriek tomu, že aktuálny trend vo vývoji hardvéru je znižovanie rozmerov a  zvyšovanie výkonu, online prediktívny regulátor je stále náročný na výpočtový výkon. Preto popri výskumoch aplikovania offline metódy prediktívneho algoritmu na FPGA hradlá, vznikla myšlienka implementácie online MPC regulátora na server - ,,do cloudu``, ako službu, kde je možné zabezpečiť takmer neobmedzený výkon a jedinou prekážkou môže byť rýchlosť sieťového pripojenia. Realizácia tejto myšlienky je implementovaná v prostredí inteligentnej budovy.