\indent 
\begin{itemize}
\item  spomenúť hlavné výhody prediktívnej metódy a jej súčasný stav.
\item  Predmetom práce je tiež definovať pojem Internet vecí (Internet of Things - IoT) ako nový typ architektúry informačných systémov a jeho špecifiká. 
\item  V praktickej časti je spojenie predošlých dvoch bodov v myšlienke regulátor ako služba (Controller as a Service - CaaS) a jej implementácia. 
\end{itemize}
 
\subsection*{Motivácia k MPC}

\subsection*{Motivácia k IoT}
Aktuálne je pojem IoT čoraz častejšie skloňovaný na konferenciách akademickej a rovnako aj komerčnej sféry. Rôzne popredné spoločnosti ako IDC, Gartner ai. zaoberajúce sa výskumnými a poradnými činnosťami v oblasti informačných a komunikačných technológií robia odhady využitia. Tvrdenie portálu www.crn.com:
,,Napriek tomu, že IoT bol vždy trochu vágny pojem pri špecifikácii obchodných partnerov a produktov, ktoré už sú na trhu, analytici predikovali, že pripojené zariadenia (connected devices) majú veľký potenciál príležitosti, ktoré budú výnosné. Júnový prieskum trhu spoločnosti IDC predikoval, že výdavky na IoT dosiahnú v roku 2020 sumu vo výške 1,7 biliónov dolárov, zatiaľ čo Gartner predpovedal, že v tom istom roku bude pripojených 21 miliárd zariadení.``\cite{IOT01} Na základe tohto a ďalších podobných článkov je teda motivácia hľadanie vhodných prípadov využitia IoT.\\
\indent Okrem toho treba zopakovať, že IoT je spojenie minimálne dvoch rozsiahlych technických odborov, neberúc do úvahy spoločenské, prírodne ani lekárske vedné odbory, ktoré tiež môžu skúmať dopad IoT na ne. Rozsiahlosť tejto problematiky je určite nepopierateľná. Druhá motivácia teda je získavanie  nadhľadu nad spleťou vznikajúcich a zanikajúcich technológií a štandardov.\\
\indent Ostatná motivácia k skúmaniu IoT je porovnanie rozdielov pri návrhu, vývoji a správe oproti klasickým čisto softvérovým informačným systémom. 
\subsection*{Motivácia k CaaS}
Napriek tomu, že aktuálny trend vo vývoji hardvéru je znižovanie rozmerov a  zvyšovanie výkonu, online prediktívny regulátor je stále náročný na výpočtový výkon. Preto popri výskumoch aplikovania offline metódy prediktívneho algoritmu na zariadenia blízke hardvérovej úrovni napríklad FPGA hradlá, vznikla myšlienka implementácie online MPC regulátora na server - ,,do cloudu``, ako službu, kde je možné zabezpečiť takmer neobmedzený výkon a jedinou prekážkou môže byť rýchlosť sieťového pripojenia. Realizácia tejto myšlienky je implementovaná v prostredí inteligentnej budovy.