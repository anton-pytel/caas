\indent V posledných rokoch môžeme sledovať veľký rozmach pripájania do siete internet, či už mobilných zariadení, počítačov nehovoriac a sú prípady pripájania už aj chladničiek. Rovnako môžeme sledovať zvýšenie dostupnosti elektronických súčiastok a lacných zariadení typu Raspberry Pi. Okrem toho vidíme kontinuálny rozvoj v oblasti návrhu a vývoja softvéru nové jazyky, nové štandardy, nové spôsoby integrácie. Všeobecne môžeme povedať, že informačné technológie idú dopredu rýchlym tempom. Výsledkom rozvoja informačných technológii a spomínaných oblastí je vznik nových typov informačných systémov s názvom internet vecí - IoT (internet of things) systémy. Aktuálne je pojem IoT čoraz častejšie skloňovaný na konferenciách akademickej a rovnako aj komerčnej sféry. Rôzne popredné spoločnosti ako IDC, Gartner ai. zaoberajúce sa výskumnými a poradnými činnosťami v oblasti informačných a komunikačných technológií robia odhady využitia. Tvrdenie portálu www.crn.com:
,,Júnový prieskum trhu spoločnosti IDC predikoval, že výdavky na IoT dosiahnú v roku 2020 sumu vo výške 1,7 biliónov dolárov, zatiaľ čo Gartner predpovedal, že v tom istom roku bude pripojených 21 miliárd zariadení.``\cite{IOT01} V oblasti regulátorov určite v percentuálnom nasadení stále prevládajú konvenčné metódy typu PID, avšak dostávajú sa do praxe aj moderné metódy ako prediktívne alebo adaptívne riadenie. Zvýšiť mieru nasadenia moderných metód riadenia je možné práve overením ich funkčnosti a potvrdením ich lepších vlastností v praxi a následnou propagáciou týchto výsledkov.\\
\indent Cieľom práce je prepojenie novovznikajúcich IoT systémov s prediktívnou metódou riadenia (MPC - model predictive controller). Ak chceme hlavný cieľ prace špecifikovať do detailu, tak ide tu o overenie zavedenia myšlienky regulátor ako služba (CaaS - controller as a service) do praxe. V práci sa overuje konkrétne online forma prediktívneho riadenia s obmedzeniami a sledovaním referenčnej hodnoty, čo je vysvetlené v prvej časti práce. Myšlienka regulátor ako služba predstavuje koncept, v ktorom je celá zložitosť výpočtu akčného zásahu regulátora, ktorú MPC prináša, na serveri a akčný člen len vykonáva vypočítaný akčný zásah riadenej veličiny. V práci sú opísané detaily tejto myšlienky, implementácie, s poukázaním na výhody a nevýhody navrhovaného prístupu. Regulátor sa v práci implementuje do reálneho IoT systému inteligentnej domácnosti. Prechod od opisu prediktívneho regulátora k jeho overeniu tvorí časť práce popisujúca trendy softvérových architektúr. V tejto časti sú vysvetlené architektúry a architektonické princípy, ktoré sú význačné pre IoT systémy a tiež detailne vysvetlený pojem IoT. Vzhľadom na to, že myšlienka CaaS by bez IoT systému nebola realizovateľná, rovnaké architektonické princípy sú použité  pri implementácii riešenia CaaS. Po opise implementácie prediktívneho regulátora do IoT systému je najdôležitejšia časť v práci a to overenie na konkrétnych fyzických zariadeniach. Overenie možnosti realizácie prediktívneho riadenia pomocou IoT architektúry je realizovaná na systéme udržiavania požadovanej hladiny intenzity osvetlenia a je použiteľný na riadenie a ovládanie iných typov technologických procesov v IoT systéme. Na konci práce je posledná časť venujúca sa rozdielom medzi klasickým softvérovým systémom a IoT systémom z viacerých uhlov pohľadu, vychádzajúc z praktických skúsenosti návrhu, vývoja a prevádzkovania IoT systému.