Hlavným cieľom predloženej dizertačnej práce bolo vypracovanie novej metodiky pre návrh, vývoj a realizáciu prediktívnych regulátorov v súlade s rozvojom moderných informačných technológií využívajúcich nové trendy, ktoré môžeme charakterizovať ako efektívne spojenie metód a algoritmov riadenia s IoT prístupmi a technológiami. Na základe tejto metodiky je v práci navrhovaný nový prístup riadenia s novou myšlienkou, kde regulátor je chápaný ako služba (Caas - Controller as a Service). Vzhľadom na náročnosť riešení úloh prediktívneho riadenia, ktoré vyžadujú zložité numerické algoritmy na optimalizáciu so zahrnutím ohraničení na riadiaci zásah, diferenciu riadiaceho zásahu a výstupnú regulovanú veličinu, bolo nutné zefektívniť takéto riešenia cez IoT systém a CaaS prístup.\\
V práci je pojem IoT zadefinovaný podloženými literárnymi zdrojmi. IoT systém je v práci podrobne opísaný a architektonické voľby návrhu systému zdôvodnené. Rovnako implementácia IoT systému a problém riadenia sa opiera o architektonické princípy, ktoré sú definované v práci a vychádzajú z najnovších trendov v informačných technológiách. Myšlienka CaaS a jej realizácia je v práci opísaná. Ide tu o vystavenie online metódy MPC regulátora prostredníctvom REST architektonického princípu, čo znamená, že bol vytvorený REST resource (zdroj) regulátor a jeho subresource (podzdroj) krok regulácie daného regulátora. Na základe tohto prístupu je vyvinutá aplikácia riadenia jednoducho aplikovateľná do iných IoT systémov. Jediná požiadavka na umožnenie regulácie procesu je, aby akčný člen vedel vytvoriť HTTP požiadavku a spracovať správu vo formáte JSON.\\
Využitie prediktívneho regulátora a jeho zabudovania do IoT bolo v práci overené na reálnej inteligentnej domácnosti, kde hlavnou úlohou MPC regulátora, bolo zabezpečenie intenzity osvetlenia. Zložité numerické riešenia boli poskytované prostredníctvom IoT Backend súčasti, ktorá je charakteristická disponovaním veľkou výpočtovou kapacitou. V práci bol vytvorený podporný program na simuláciu MPC regulátora, ktorý bol využitý pri realizácii CaaS na porovnávanie výsledkov reálneho systému a simulácie. Pri tomto riešení sa ukázali niektoré nedostatky v ohraničení možnej periódy vzorkovania a v kritických procesoch, pri ktorých je potrebné získanie odozvy v kratšom čase, ako je čas potrebný na sieťovú komunikáciu. Avšak výhody navrhovaného prístupu sú veľmi významné a ukázalo sa, že aj zložité výpočty je možné realizovať takmer v reálnom čase a že je možné vytvoriť ľubovoľný počet inštancií regulátorov a riadiť tak všetky procesy v IoT systéme bez ďalších nákladov, čo potvrdzuje uvedené výhody.\\
Výsledky dizertačnej práce možno využiť pri riešení nových výskumných úloh zameraných na moderné metódy automatického riadenia v integrácii s novými informačnými technológiami využívajúce IoT a Big Data.
