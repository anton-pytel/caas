Hlavným cieľom práce bolo aplikovať prediktívny regulátor do IoT systému, z čoho ako je už v úvode napísane vznikla myšlienka CaaS - regulátor ako služba, keďže MPC je metóda riadenia s pomerne náročnými požiadavkami na matematické výpočty, čím sa cieľ rozšíril o overenie tejto myšlienky. S aplikáciou prediktívneho riadenia do IoT systému a overovania myšlienky CaaS bol nevyhnutý návrh, implementácia, prevádzka a samozrejme aj definícia IoT systému, čo boli ďalšie činnosti rozširujúce cieľ práce. Vysvetlenie a odôvodnenie použitých architektonických princípov pri implementácii sme pokladali za povinnosť uviesť do práce. Rovnako postrehy z prevádzky IoT systému sme vyhodnotili ako dôležité v práci spomenúť, to už však len ako vedľajší cieľ práce.\\
\indent Cieľ aplikovania prediktívneho regulátora do IoT systému považujeme za splnený. Rovnako overenie, či je možné MPC do IoT systému zaviesť konceptom CaaS, považujeme za splnené, keďže bol v práci navrhnutý, implementovaný a prevádzkovaný IoT systém v inteligentnej domácnosti, v ktorom sa pomocou MPC regulátora udržiavala hladina žiadanej intenzity osvetlenia a výpočty regulátora náročné na výpočtový výkon boli poskytované prostredníctvom IoT Backend súčasti, ktorá je charakteristická disponovaním veľkej kapacity výpočtového výkonu. Vyhodnotenie kvality riadenia naznačilo nedostatky tohto prístupu práve v potenciálnej obmedzenosti periódy vzorkovania regulátora a pri kritických procesoch závislosťou na pripojení 24/7. Avšak výhody, že je možné aj zložité výpočty realizovať v kvázi reálnom čase a že je možné vytvoriť ľubovoľný počet inštancií regulátorov a riadiť tak všetky procesy v danom IoT systéme, bez dodatočných nákladov, sú nesporné. Cieľ definície, návrhu, implementácie a prevádzky IoT systému považujeme za splený. V práci je pojem IoT zadefinovaný podloženými zdrojmi. IoT systém bol v práci podrobne popísaný a architektonické voľby zdôvodnené rovnako ako implementácia IoT systému, opierajúce sa o architektonické princípy definované v práci na základe zdrojov. Vedľajší cieľ nahliadnutia do prevádzky IoT systému tiež považujeme za splnený, pretože práca prináša pohľad na to, aké činnosti sú pre jednotlivé fázy životného cyklu IoT systému potrebné, aké kompetencie respektíve oblasti zájmu je potrebné sledovať, aby bola udržiavaná kvalita IoT systémov, či už novo vznikajúcich alebo existujúcich a pridaná hodnota IoT systémov nebola znižovaná ich nespoľahlivosťou, nedostatočným zabezpečením, či dokonca nebezpečnosťou prevádzky.
