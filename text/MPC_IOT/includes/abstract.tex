Hlavná myšlienka dizertačnej práce je aplikovať moderné metódy riadenia do IoT systému. Práca sa v prvej časti venuje matematickému opisu prediktívneho riadenia. Ďalej opisu podporného programu na simuláciu prediktívneho riadenia, ktorý bol vytvorený v prvej časti štúdia. Podporný program je vytvorený v prostredí Matlab-Guide a umožňuje testovať, ladiť a konfigurovať prediktívne algoritmy riadenia. 
Po objasnení prediktívneho riadenia sa práca zaoberá využitím znalostí z oblasti softvérovej architektúry na definovanie IoT systému, trendy v tejto oblasti a hlavne uskutočnenie hlavnej myšlienky a teda nájdenie optimálneho využitia IoT komponentov na umiestnenie prediktívneho regulátora do nich. Takto vzniká myšlienka regulátor ako služba (Controller as a service - CaaS), ktorá je v práci vysvetlená. Popísané sú výhody a nevýhody tohto prístupu riadenia. Myšlienka je overená v praxi na IoT systéme, ktorého súčasti sú v práci podrobne popísané. Po overení CaaS myšlienky reálnou implementáciou, je v práci ako vedľajšia myšlienka zhrnutie rozdielov, ktoré  nové IoT systémy prinášajú oproti klasickým, čisto softvérovým systémom. Tieto postrehy vychádzajú z návrhu, implementácie súčastí a  prevádzkovania IoT systému v praxi. V práci je využitý IoT systém inteligentnej budovy.
