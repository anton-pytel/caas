Hlavná myšlienka dizertačnej práce je modifikácia moderných, existujúcich metód automatického riadenia, tak aby boli implementovateľné do nových foriem realizácie, ktoré využívajú IoT prostredie. Dizertačná práca je zložená z dvoch základných častí a to teoretickej a praktickej. Teoretická práca sa opiera o opis moderných metód riadenia, konkrétne prediktívneho riadenia a ich modifikácií, tak aby boli aplikovateľné do reality v integrácií s modernými informačnými technológiami založenými na IoT architektúre. Praktická časť dizertačnej práce je založená na vývoji všeobecného programového systému, ktorý umožňuje modelovať, simulovať a konfigurovať prediktívne algoritmy riadenia. Významnou súčasťou dizertačnej práce je návrh a realizácia softvérového systému, ktorého nosnou časťou je IoT architektúra. V rámci tejto architektúry vzniká myšlienka prevádzkovať regulátor ako službu (Controller as a service - CaaS). Takto navrhovaná metodika riadenia je overená v praxi na IoT systéme, ktorý predstavuje inteligentnú domácnosť. V  dizertačnej práci je porovnaná nová metodika s klasickým prístupom k informačnému systému pre riadenie. 