Písomná práca k dizertačnej skúške je zameraná v prvom rade na odbor automatizácia, konkrétne na opis vybraných metód a algoritmov prediktívneho riadenia lineárnych dynamických systémov. Na základe teoretických princípov metód prediktívneho riadenia z domácej a zahraničnej literatúry bol v priebehu prvej časti doktorandského štúdia vyvinutý všeobecný podporný programový systém, pre účely výučby a výskumu v predmetnej oblasti.  Podporný programový systém je vytvorený v prostredí Matlab-Guide a umožňuje testovať, ladiť a konfigurovať rôzne prediktívne algoritmy riadenia. Všeobecný programový systém bol testovaný pre rôzne typy priemyselných procesov (tepelné-optické procesy, DC motory, riadenie turbíny apod.).\\
\indent V druhom rade sa zameriava na odbor informatika. Práca rozoberá možnosti architektúr softvéru a ich prienik s odborom automatizácia. S využitím znalostí z oboch odborov je popísaný typ architektúry nazývaný Internet vecí (Internet of Things - IoT). Sú taktiež definované jeho vlastnosti, prípady použitia a ich variácie. V tejto časti sú analyzované aj rozdiely a zhody v návrhu, vývoji a spravovaní existujúcich aplikácií voči IoT systémom.\\
\indent V poslednom rade je v práci popísaný konkrétny IoT systém inteligentnej budovy a všetky jej súčasti. Myšlienka regulátor ako služba (Controller as a service) a jej implementácia v uvedenom IoT systéme. Konkrétne ide o implementáciu online prediktívneho regulátora.