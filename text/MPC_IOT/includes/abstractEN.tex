The main aim of this dissertation thesis is modification of modern, existing methods of automatic control, so that they could be implemented into new forms of realization, which use IoT environment. Dissertation thesis consist of two basic parts, namely theoretical and practical. Theoretical work is based on description of modern control methods, specifically predictive control and its modification, so it could be applied into reality with integration of modern informational technologies based on IoT architecture. Practical part of the dissertation thesis is based on development of general application system, that enables user to design, simulate and configure predictive control algorithm. Significant part of the dissertation thesis is design and realization of software system, which is based on the IoT architecture. In the context of this architecture, new idea of providing controller as a service (CaaS) evolves. Proposed methodology of control is proven in the reality in IoT system, which is smart home. Comparison of new methodology and classical IT system for control is in the dissertation thesis.